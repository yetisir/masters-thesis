%======================================================================
\chapter{Conclusions and Future Considerations}
%======================================================================

A summary of the main conclusions from the development of the up-scaling framework as well as the implementation and testing of the framework are presented here. These conclusions represent a successful completion of the research objectives for the thesis. That being said, there are substantial limitations to this research. A series of recommendations are provided to address some of these limitations and to provide guidance on how to extend this research.

%----------------------------------------------------------------------
\section{Conclusions}
%----------------------------------------------------------------------
A multi-scale framework for up-scaling DEM simulations has been developed to address the computational demands of simulating microscale phenomenon in a macroscale domain in the context of NFR. Up-scaling is achieved by matching homogenized stress-strain curves from REV-scale DEM simulations to single element continuum models using PSO and LMA optimization algorithms. A Drucker-Prager plasticity model with ductile damage is implemented in the CDM model to empirically capture the effect of the degradation (damage) of the NFR as deformation takes place.

\subsection*{1. Deformable DEM Homogenization}

Homogenization algorithms were developed for homogenizing DEM simulations with deformable bodies to assess the spatially averaged stress-strain behaviour of the REV from the microscale displacements, strains, and stresses. In this homogenization process, the resultant inter-block contact forces and block displacement from the DEM simulations are converted to average stresses and strains. To apply the homogenization algorithms, a method of automatically assessing a suitable REV given a sufficiently large domain was developed. These algorithms were implemented in Python\textsuperscript{TM} as the HODS software, which was used as a module for MOUSE.

\subsection*{2. Parameterization Methodology}

Two examples of the parameterization methodology are presented. Here, the key parameters required to capture the salient features of the model are isolated in order to be able to run the parameter estimation algorithms effectively. the main aspect of this parameterization methodology is the functional assumptions of the hardening/softening and damage evolution functions. In addition, the parameters are rewritten in term of physically meaningful parameters to provide more insight into the mechanics. The Drucker-Prager model with ductile damage is shown to be a reasonable CDM model approach to represent NFR in a continuum context, including effects of pressure dependent yield and the triaxiality based damage initiation criterion. Compared to a full DEM simulation, the CDM model shows a good fit pre-damage, but is unable to emulate the subtle post-yield oscillations arising from non-continuous yielding in the NFR.

\subsection*{3. Up-Scaling Framework}

MOUSE software was created and written in Python\textsuperscript{TM} to provide an implementation of the up-scaling framework presented in this thesis using in house and third-party software modules. The software itself provides a platform through which the four software elements of the up-scaling framework (DEM module, homogenization module, parameter estimation module, and macroscale module) can communicate with each other. The communication is facilitated by MOUSE through modules which wrap the third party software in such a way that the I/O routines to and from the modules are performed in a consistent capacity regardless of the third party software being used. A consistent set of data protocols were developed for the modules to effectively transfer data between them.

\subsection*{4. Framework Verification}

The parameter estimation module was tested and yielded an appropriate parameter set that both matched the DEM data and provided realistic parameters. Additionally, the Drucker-Prager model with ductile damage was found to provide a far superior fit than the damage plasticity model for quasi-brittle materials. Most importantly, the DNS of the slope stability analysis showed that with this up-scaling framework, very significant computational gains can be had with an acceptable error. Very comparable results ($<5\%$ error) to full DEM solutions were obtained with the CDM method but required two orders of magnitude less computational time. The computational demands were again able to be reduced by another order of magnitude by using a selectively refined continuum mesh at the locations in the domain with high stress gradients.

%----------------------------------------------------------------------
\section{Recommendations}
%----------------------------------------------------------------------

The main limitation of the presented up-scaling implementation is the macroscale constitutive model module. In the current ABAQUS\textsuperscript{TM} module, the constitutive models consider only isotropic behaviour. In addition, the model does not consider the effects of pore pressure in the rock mass or fluid flow in any capacity. Though the isotropic assumptions for the elasto-plastic constitutive relationships are likely sufficiently accurate, future implementations of the macroscale constitutive model should consider anisotropic damage behaviour, as anisotropic implementations are found to be completely insufficient. In the case of the Drucker-Prager model with ductile damage, the exponential Johnson-Cook triaxiality based damage initiation criterion provides an excellent fit for monotonic loading, but does not model cyclic loading well. Here, it would be ideal for the cyclic loading capacity of the damage plasticity model for quasi-brittle materials to be incorporated as well. Ultimately, the available damage material subroutines in ABAQUS\textsuperscript{TM} are insufficient for the key physical characteristics in the system to be captured. As such, a custom anisotropic damage implementation is recommended for the macroscale constitutive model.

Retrospectively, the functional form of the hardening curve is overly complex. Though it is often necessary to model the softening of the material in the plasticity model, with CDM the damage can implicitly model the softening behaviour. Here, it is observed that for the Barcelona model used for the hardening/softening curve, only the hardening portion of the curve is ever used. As such, for future implementations of the plasticity hardening functions, a simpler exponential function could be applied which would also have the benefit of decreasing the number of parameters that need to be estimated, leading to more consistent solutions and faster convergence of the optimization algorithms. 

Furthermore, it is speculated by the author that portions of the parameterization methodology could be modified to yield more consistent solutions. In the parameterization formulations presented in this thesis, too much emphasis was placed on creating physically meaningful parameters rather than numerically consistent parameters. This inconsistency is the case with certain paired parameters if one of the parameters is highly sensitive.

Additionally, effectively searching a 11+ dimensional parameter space is computationally expensive. By dividing the problem and exploiting features of the curves and constitutive models, it may be possible to increase the convergence rate and effectively get a better, faster solution. Instead of searching the entire parameter space, it is possible to split the parameters into groups (e.g. pre-damage and post damage). Here, the damage parameters don't actually affect the plasticity calculations until damage is initiated. As such, the plasticity parameters can be estimated using the pre-damage curve and the damage parameters can subsequently be estimated using the post-damage curve.

A more rigorous examination of other available optimization routines and associated optimization parameters would be another way to potentially reduce the computational cost and increase the accuracy of the parameter estimation process. PSO and LMA were used in this research, but Dynamically Dimensioned Search (DDS), 
Real-Coded Genetic Algorithm (RGA) and Simulated Annealing (SA) were also briefly investigated. As previously stated, rigorously assessing the most effective algorithm was not a priority of this research, but the PSO + LMA combination was chosen for it's simplicity and effectiveness. The other algorithms, if properly applied may provide a faster and more accurate solution.
	
For this up-scaling methodology to be more accurate, 3D DEM simulations are required to capture accurate physical responses of these complex systems. In addition to modifying the DEM simulations, the associated homogenization algorithms would have to be modified to provide the 3D stress and strain tensors. 

Hydro-mechanically coupled DEM simulations are also an important consideration for more accurate simulations. Here, the homogenization module should be modified to assess the homogenized fluid properties such as pore pressure and flow velocity vectors and the macroscale model needs to be modified to simulate poroelastic physics.

Ultimately, rigorous validation studies should be done with real-world applications to show the viability of this approach. Incorporating some of the above reccomendations will allow this up-scaling framework to validly be applied to complex geomechanical problems in order to avoid the computational demands of DEM modelling. 