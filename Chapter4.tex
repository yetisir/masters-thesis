%======================================================================
\chapter{Software Implementation}
%======================================================================

Python notation

\section{Base Module Class}

A base module class is implemented to provide a framework containing required methods and attributes for the UpFrac modules to inherit.

\subsection{Attributes}

\begin{lstlisting}[frame=single]   
    self.program = program
    self.parameters = parameters
    
    self.suppressText = suppressText
    self.suppressErrors = suppressErrors
    
    topDirectory = os.path.realpath(os.path.dirname(os.path.dirname(__file__)))
    dataDirectory =  os.path.join(topDirectory, 'Data')
    self.binaryDirectory = os.path.join(dataDirectory, 'Binary')
    self.textDirectory = os.path.join(dataDirectory, 'Text')
\end{lstlisting}


\subsection{Defined Methods}
Discontinuous systems are characterized by the existence of discontinuities that separate discrete domains within the system. In order to effectively model a discontinuous system, it is necessary to represent two distinct types of mechanical behaviour: the behaviour of the discontinuities and the behaviour of the solid material.

\begin{lstlisting}[frame=single]
def __init__(self, program, parameters={}, suppressText=False, suppressErrors=True):
\end{lstlisting}

Discontinuous systems are characterized by the existence of discontinuities that separate discrete domains within the system. In order to effectively model a discontinuous system, it is necessary to represent two distinct types of mechanical behaviour: the behaviour of the discontinuities and the behaviour of the solid material.

\begin{lstlisting}[frame=single]
def printText(self, text):

def printTitle(self, title):

def printSection(self, section):

def printStatus(self, status):

def printDone(self):

def clearScreen(self):

def printErrors(self, error):
\end{lstlisting}

Discontinuous systems are characterized by the existence of discontinuities that separate discrete domains within the system. In order to effectively model a discontinuous system, it is necessary to represent two distinct types of mechanical behaviour: the behaviour of the discontinuities and the behaviour of the solid material.

\begin{lstlisting}[frame=single]        
def saveData(self, data):

def loadData(self):
\end{lstlisting}


\begin{lstlisting}[frame=single]    
def updateParameters(self, parameters):
\end{lstlisting}


\begin{lstlisting}[frame=single]
def commandLineArguments(self):
\end{lstlisting}


\begin{lstlisting}[frame=single]    
def run(self):
\end{lstlisting}

\subsection{Undefined Methods}

\begin{lstlisting}[frame=single]    
def createArgumentParser(self):
    
def parseArguments(self):

def inputFileName(self):
        
def outputFileName(self):
    
def setParameters(self):
\end{lstlisting}


\section{DEM Module}
\subsection{Base DEM Module Class}
\subsection{UDEC\textsuperscript{TM} Module}
\section{Homogenization Module}
\subsection{Base Homogenization Module Class}
\subsection{HODS\textsuperscript{TM} Module}
\section{Parameter Estimation Module}
\subsection{Base Parameter Estimation Module Class}
\subsection{OSTRICH\textsuperscript{TM} Module}
\section{Macroscale Module}
\subsection{Base Macroscale Module Class}
\subsection{ABAQUS\textsuperscript{TM} Module}


\section{Data Architechture}
This section aims to explain the data structures used to store and transfer data between the modules in the up-scaling framework presented in this thesis.  Three fundamental data structures are used for the storage of data in conjunction with each other: hash tables, lists, and arrays. Using these three structures, three distinct types of data in the up-scaling framework are used to transfer between the modules in a consistent manner: constitutive parameters, DEM data, and continuum data.

\subsubsection*{Python Lists}
The list in Python\textsuperscript{TM} is one of the most versatile data structures. The list treats the data as a sequence such that each element in the list is assigned a number referring to its position or index.  The implementation of the list structure in python contains a number of unique features. The main feature of python lists is that the elements within the lists are not required to be of the same data type. Additionally, these lists are mutable, allowing for more advanced manipulation of the data structure. Because of this flexibility, the list structure is slow and unsuitable for large data sets. 

\subsubsection*{Numpy Arrays}
Numpy arrays are a specific implementation of a python list which are optimized for numerical operations on large datasets. Here the numpy arrays are much more compact by using a sequence of uniform values, rather than a sequence of pointers as is the case for the list structure.

\subsubsection*{Python Dictionaries (Hash Tables)}
In general, a hash table is a data structure that maps keys to values. The implementation of hash tables in Python\textsuperscript{TM}, are referred to as dictionaries. Similar to a python list, a python dictionary is a mutable structure that does not require the elements to be of the same type. However, instead of accessing the element value through an index argument, the data is accessed through a key argument. This unordered structure is usefull for storing non-sequential data.

\subsection{Data Storage Structures}


\subsubsection*{Hash Tables for Constitutive Parameters}

Constitutive parameters are required as an input to the DEM and macroscale modules as well as an output from the parameter estimation module. Here, the constutive parameters are always stored in a Python\textsuperscript{TM} dictionary with the constitutive parameter name as the key. To access the value of a specific constitutive parameter in a dictionary named 'constitutiveParameters', the string corresponding to the 

\begin{lstlisting}[frame=single]
parameterValue = constitutiveParameters[parameterName]
\end{lstlisting}

For example, accessing the value of the elastic modulus from a hash table named 'contitutiveParameters' is performed as follows:

\begin{lstlisting}[frame=single]
elasticModulus = constitutiveParameters['elasticModulus']
\end{lstlisting}


\subsubsection*{Nested Hash Tables for DEM Data}
The raw DEM Data is considered to be comprised of six distinct types: block data, contact data, corner data, domain data, grid point data, and zone data. Here, each block, contact, corner, domain, grid point, and zone is assigned a unique 7 digit numeric identifier (assuming here that the number of components in the system does not exceed 10 million) by which the associated data can be accessed. The same identifier may be repeated for different data types.

To allow for convenient access to the data, the DEM data is stored in six distinct nested hash tables corresponding to each of the DEM data types.  Each DEM data hash table has three levels of nesting. The first level keys are the simulation times, which returns the second level of hash tables. The second level keys are the component identifiers, which returns a third level hash table. In this third level, the component attirbutes can be accessed using the attirbute name as the key. In general, accessing an attribute of a specific component with a given ID at a specific time from a hash table called demDataHash is done as follows:


\begin{lstlisting}[frame=single]
componentAttributeValue = demDataHash[time][ID][attribute]
\end{lstlisting}

For example, accessing the list of grid points associatied with block ID 2543465 at a simulation time of 1 is performed as follows:

\begin{lstlisting}[frame=single]
gridPoints = blockData[1.0][2543465]['gridPoints']
\end{lstlisting}


block stuff Table \ref{tab3}

\begin{table}[!htb]
\centering
\caption{{Block data attributes in third level hash}}
\label{tab1}
\begin{tabularx}{\textwidth}{@{}Yp{0.75\textwidth}@{}}
\toprule
\textbf{Parameter} & \textbf{Description (Data Type)}                            \\ \midrule
x                  & X coordinate of block centroid (float)                      \\
y                  & Y coordinate of block centroid (float)                      \\
xForce             & Resultant forces at block centroid (float)                  \\
yForce             & Resultant forces at block centroid (float)                  \\
corners            & Corners associated with this block (list of corner IDs)     \\
zones              & Zones associated with this block (list of zone IDs)         \\
gridPoints         & Grid points associated with this block (list of corner IDs) \\ \bottomrule
\end{tabularx}
\end{table}

\begin{table}[!htb]
\centering
\caption{{Contact data attributes in third level hash}}
\label{tab2}
\begin{tabularx}{\textwidth}{@{}Yp{0.75\textwidth}@{}}
\toprule
\textbf{Parameter} & \textbf{Description (Data Type)}                          \\ \midrule
x                  & X coordinate of contact point (float)                     \\
y                  & Y coordinate of contact point (float)                     \\
length             & Length associated with contact point (float)              \\
flowRate           & Fluid flow rate through contact (float)                   \\
nForce             & Resultant normal force on contact (float)                 \\
sForce             & Resultant shear force on contact (float)                  \\
xCosine            & X component of contact normal cosine (float)              \\
yCosine            & Y component of contact normal cosine (float)              \\
blocks             & Blocks associated with this contact (list of block IDs)   \\
domains            & Domains associated with this contact (list of domain IDs) \\
corners            & Corners associated with this contact (list of corner IDs) \\ \bottomrule
\end{tabularx}
\end{table}

\begin{table}[!htb]
\centering
\caption{Corner data attributes in third level hash}
\label{tab3}
\begin{tabularx}{\textwidth}{@{}Yp{0.75\textwidth}@{}}
\toprule
\textbf{Parameter} & \textbf{Description (Data Type)}                                 \\ \midrule
gridPoint          & Grid points associated with this corner (list of grid point IDs) \\ \bottomrule
\end{tabularx}
\end{table}

\begin{table}[!htb]
\centering
\caption{Domain data attributes in third level hash}
\label{tab4}
\begin{tabularx}{\textwidth}{@{}Yp{0.75\textwidth}@{}}
\toprule
\textbf{Parameter} & \textbf{Description (Data Type)}            \\ \midrule
x       & X coordinate of domain centroid (float)     \\
y       & Y coordinate of domain centroid (float)     \\
area               & Area of domain (float)                      \\
porePressure      & Average pore pressure in the domain (float)\\ \bottomrule
\end{tabularx}
\end{table}

\begin{table}[!htb]
\centering
\caption{Gridpoint data attributes in third level hash}
\label{tab5}
\begin{tabularx}{\textwidth}{@{}Yp{0.75\textwidth}@{}}
\toprule
\textbf{Parameter} & \textbf{Description (Data Type)}                             \\ \midrule
x       & X coordinate of grid point (float)                           \\
y       & Y coordinate of grid point (float)                           \\
xDisp     & X displacement of grid point (float)                         \\
yDisp     & Y displacement of grid point (float)                         \\
xForce            & Resultant X force on grid point (float)                      \\
yForce            & Resultant Y force on grid point (float)                      \\
xVel         & X velocity of grid point (float)                             \\
yVel         & Y velocity of grid point (float)                             \\
block              & Block associated with this grid point (block ID)   \\
corner             & Corner associated with this grid point (corner ID)\\ \bottomrule
\end{tabularx}
\end{table}

\begin{table}[!htb]
\centering
\caption{{My caption}}
\label{tab6}
\begin{tabularx}{\textwidth}{@{}Yp{0.75\textwidth}@{}}
\toprule
\textbf{Parameter} & \textbf{Description (Data Type)}                                \\ \midrule
S11                & X component of stress in the zone (float)                       \\
S22                & Y component of stress in the zone (float)                       \\
S12                & Corresponding shear stress in the zone (float)                  \\
block              & Block associated with this zone (block ID)                      \\
gridPoints         & Grid points associated with this grid point (list of block IDs)\\ \bottomrule
\end{tabularx}
\end{table}




\subsubsection*{Nested List of Arrays for Continuum Data}

% \begin{equation}

% \left[ \begin{array}{cc} b & c \\ e & f \\ h & i \end{array}\right]

% \end{equation}

\subsection{Binary Serialization}




\section{HODS\textsuperscript{TM} Homogenization}
\subsection{Class Attributes}
\subsection{Class Boundary Methods}
\subsection{Class Manipulation Methods}
\subsection{Class Homogenization Methods}

