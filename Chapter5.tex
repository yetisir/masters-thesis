%======================================================================
\chapter{Verification and Application}
%======================================================================
In this section, two-dimensional DEM models are used to demonstrate the effectiveness of the up-scaling methodology. While the results presented leave a positive impression, future verification using three-dimensional models is desirable. The framework is validated using three tests: 1) The homogenized stress and strain behaviour obtained from the DEM microscale response are compared to that of the macroscale response. This test verifies the effectiveness of the parameter estimation module and the ability of the chosen macroscale constitutive model to capture the salient features of NFR behaviour. 2) The homogenization and parameter estimation algorithms are rerun using the same data, but with different REV sizes to investigate the REV size effect has on the resultant parameter set. 3) Slope stability analyses carried out by both Direct Numerical Simulations (DNS) with a DEM model and with an up-scaled macroscale model are compared. This last test verifies the whole up-scaling methodology.

The up-scaling is conducted by running a series of four quasi-static DEM 'triaxial' compression tests under different confining stresses. These are not true triaxial tests as simulations are in 2D, but illustrate the method regardless. Algorithms are rerun using different REV sizes to determine an appropriate REV size. In a macroscopically homogenous domain, as the REV size increases, the parameter values will converge to a single value, when the REV is too small, local heterogeneities induce a variance into the optimal parameter set.

\section{DEM Simulations}

The DEM models used consist of a pseudo-random isotropic fracture network defined by a Voronoi tessellation. The average block size is specified to be 0.5m using 20 iterations of Lloyd's method \citep{Lloyd_1982} in order to achieve an even size distribution. A 10m x 10m domain was determined to be sufficiently large to represent the rock mass behaviour as an REV. 

A Mohr-Coulomb plasticity model was used as the constitutive model to describe the plastic behaviour of the intact  (deformable blocks) and the joint (natural fracture) behaviour was governed by a Coulomb area slip model. The parameters for the rock and joints summarized in Table \ref{tab:demProp} are representative of a fractured granitic rock mass. The joints are relatively weak compared to the blocks, so the blocks behave mostly elastically. 

\begin{table}[!htb]
\centering
\caption{{Rock and joint properties for DEM Simulations}}
\label{tab:demProp}
\begin{tabularx}{\textwidth}{@{}YYY@{}}
\toprule
\textbf{Property Type} & \textbf{Property} & \textbf{Value} \\ \midrule
\multirow{7}{*}{Rock}  & Young's Modulus   & $65 GPa$       \\
                       & Poisson's Ratio   & $0.2$          \\
                       & Density           & $2.7 g/cm^3$   \\
                       & Friction Angle    & $51^{\circ}$   \\
                       & DilationAngle     & $0^{\circ}$    \\
                       & Cohesion          & $55.1 MPa$     \\
                       & Tensile Strength  & $11.7 MPa$     \\ \cmidrule(r){1-1}
\multirow{6}{*}{Joint} & Friction Angle    & $32^{\circ}$   \\
                       & Dilation Angle    & $5^{\circ}$    \\
                       & Cohesion          & $100 kPa$      \\
                       & Tensile Strength  & $100 kPa$      \\
                       & Normal Stiffness  & $10 GPa/m$     \\
                       & Shear Stiffness   & $1 GPa/m$      \\ \bottomrule
\end{tabularx}

\end{table}


The blocks are meshed with linear three-node triangular plane strain finite difference elements with an average side length of 0.5m. This discretization yielded 5-10 zones within each block. A rounding length of 10\% of the average block edge length (0.05m) is applied to the blocks to prevent numerical instabilities in the contact algorithm. Quasi-static analysis is obtained through dynamic relaxation, in which the dynamic equations are integrated in time using velocity-proportional viscous damping and mass scaling. State data of the model is collected at 50 evenly spaced intervals. 

The quasi-static loading of the DEM simulations is intended to imitate triaxial laboratory tests, so a constant confining stress was applied on the lateral boundaries of the DEM model. Loading is achieved by applying vertical displacements to the top boundary while fixing the bottom boundary, compressing the model to a vertical strain of $5\%$ for four confining horizontal stresses: $0.5MPa$, $1MPa$, $2MPa$, and $4MPa$. These load paths capture key physical phenomena including the pressure dependent yield of the NFR, hardening, and the dependence of damage initiation on the triaxiality. 

%Second, triaxial compressive strength test at a strain rate of $0.05\%/s$ for $50s$ followed by a tensile strain rate of $0.05\%/s$ for $50s$ to observe the stiffness degradation and elastic recovery response of the DEM simulations during unloading. Here, even though the extended Drucker-Prager model is formulated for monotonic loading, the cyclic loading capacity of this model was investigated. The idea with these simulations was to strain the model past the yield point in order to investigate the post-yield behavior, but not strain the model to failure such that it loses all of its strength.



\section{Verification of the Parameter Estimation Module}

Using a PSO algorithm followed by an LMA optimization, the Drucker-Prager plasticity model with ductile damage is then fitted to the homogenized DEM simulation data in order to obtain an optimal parameter set. Each simulation is fit to 50 points defining the homogenized stress-strain curve resulting in a total of 200 data points for all four DEM simulations at different confining stresses. The PSO algorithm uses a swarm size of 24 for 100 generations which is found to be sufficient to converge to a consistent solution. 

Here, the CDM model is confined laterally by the homogenized horizontal DEM stress and vertical displacements are prescribed by the homogenized vertical DEM strain with the parameter estimation algorithms programmed to match the horizontal strain and the veritcal stress. Because of the large variation in observation magnitudes (between stress/strain and from different confining stresses), each curve is weighted with a normalization factor to prevent the large stress values from dominating parameter estimation. In addition, a linear weighting scheme is applied to each curve to give larger influence to the loading section and lesser influence to the post-damage section.

Parameter bounding limits are required by the optimization algorithms in order to limit the search space. These limits are chosen based on two criteria: physical limitations and numerical stability. If there exist physical limitations that prevent parameters from exceeding certain values or if there exists a range of realistic values that the parameter should not deviate from, then those physical limitations are specified as the bounds. In other cases, the parameter bounds come from numerical limitations such that beyond a certain capacity, certain parameter values would cause the simulations to become unstable. In these cases, a combination of the two bounding methods is used. The specified bounding limits for each parameter results can be seen in Table \ref{tab:paramDrucker}.

\begin{table}[!htb]
\centering
\caption{{Parameter estimation results for Drucker-Prager model with ductile damage}}
\label{tab:paramDrucker}
\begin{tabulary}{\textwidth}{@{}cCcCCC@{}}
\toprule
\textbf{Parameter}                 & \textbf{Symbol}                  & \textbf{Units} & \textbf{Lower Bound} & \textbf{Upper Bound} & \textbf{Optimum} \\ \midrule
Young's Modulus                    & $E$                              & $GPa$          & $1$                                                             & $25$                                                            & $1.8$                                                             \\
Poisson's Ratio                    & $\nu$                            &                & $0.1$                                                           & $0.4$                                                           & $0.15$                                                            \\
Dilation Angle                     & $\psi$                           & $^{\circ}$     & $5$                                                             & $15$                                                            & $22$                                                              \\
Flow Stress Ratio                  & $K$                              &                & $0.78$                                                          & $1$                                                             & $0.81$                                                            \\
Friction Angle                     & $\beta$                          & $^{\circ}$     & $45$                                                            & $60$                                                            & $56$                                                              \\
Initial Compressive Yield Strength & $\sigma_c^{iy}$                  & $kPa$          & $1$                                                             & $100$                                                           & $52$                                                              \\
Peak Compressive Yield Strength    & $\sigma_c^{p}$                   & $MPa$          & $0.5$                                                           & $5$                                                             & $3.1$                                                             \\
Strain at Peak Compressive Yield   & $\epsilon_c^{p}$                 & $\%$           & $0.5$                                                           & $5$                                                             & $1.7$                                                             \\
Yield Strain at -0.5 Triaxiality   & $\bar{\epsilon}^{pl}_{f_{-0.5}}$ & $\%$           & $0.01$                                                          & $0.1$                                                           & $0.0078$                                                          \\
Yield Strain at -0.6 Triaxiality   & $\bar{\epsilon}^{pl}_{f_{-0.6}}$ & $\%$           & $0.1$                                                           & $10$                                                            & $0.30$                                                            \\
Plastic Displacement at Failure    & $\bar{u}^{pl}_f$                 & $m$            & $0.01$                                                          & $1$                                                             & $0.12$                                                            \\ \bottomrule
\end{tabulary}
\end{table}

The stress-strain curves from the DEM simulations used for the parameter estimation and the stress-strain curves of the CDM simulations using the optimal parameter set are presented in Figure \ref{fig:fitted1}. The CDM fit is good with a Root-Mean-Square Error (RMSE) of $1.03MPa$ and the pressure dependent yield function works well with this model as the error is not biased to curves of a certain confining stress. This fit implies a strong likelihood that the model will be valid under confining stresses outside of the range fitted. Also, the damage initiation points at the peak of the curve are well correlated and indicate that the triaxiality based damage initiation criterion is a good model for this problem. The majority of the error in the curves is found in the post-yield behaviour. This error results from limitations in the continuum constitutive model because the post-yield behaviour of the DEM simulations is discontinuous in nature (stick-slip response). The CDM model cannot accommodate for such oscillations and thus represents the post-yield response as an average. 


\begin{figure}[!htb]
\begin{center}
\includegraphics[width=0.7\textwidth]{figures/druckerDamage_voronoiGranite/voronoiGranite_druckerDamage_lastFrame_dir-2}
\caption{{\label{fig:fitted1} Axial Stress-Strain curves of the monotonically loaded DEM simulations used for estimating the CDM parameter set under different confining stresses.%
}}
\end{center}
\end{figure}

The optimal parameter set in Table \ref{tab:paramDrucker} represents the constitutive response of the rock mass. As expected, the elastic modulus of the rock mass ($1.9 GPa$) is substantially less than the elastic modulus of the intact rock ($65 GPA$) because of yielding in the joints. Additionally, Poisson's ratio of the rock mass ($0.15$) is less than Poisson's ratio of the intact rock ($0.2$) because of the compliance of the joints before yield, which limits the lateral strain. After yielding however, substantial lateral strain is observed because of dilation of the joints, resulting in a large dilation angle ($22^\circ$). This dilation response of the rock mass is larger than the the prescribed joint dilation ($5^\circ$) because of block rotation and geometry.

There are additional minor sources of error from the homogenization algorithms that do not manifest themselves in this fitted relationship.  In addition, if the REV is too small, it introduces it's error in the DEM data rather than in the fitted response. Furthermore, the global fitting algorithms are not completely exhaustive, so it is possible they do not find the actual globally optimal parameter set, potentially leading to some error. With the given PSO parameters, up to 2400 sets of simulations are conducted for the global parameter estimation, and successive fitting operations tend to give results within $1\%$ deviation. This consistency and large search domain give confidence that the estimated parameter set is the globally optimal set. 

In addition to the loading response under the specified confining stresses, DEM simulations under confining stresses of $3MPa$, $6MPa$, $8MPa$ and $10MPa$ are compared to the the CDM model using the previously estimated parameter set to see how well the constitutive behaviour is captured (Figure \ref{fig:fitted2}). These simulations demonstrate the interpolative ($3MPa$) and extrapolative ($6MPa$, $8MPa$ and $10MPa$) capacity of the fitted parameter set, and indeed a strong fit is obtained (RMSE of $2.83MPa$) for all confining stresses, with the error being more prominent for larger degrees of strain.

\begin{figure}[!htb]
\begin{center}
\includegraphics[width=0.7\textwidth]{figures/fittedCurves/voronoiGraniteVerify_druckerDamage_lastFrame_dir-2}
\caption{{\label{fig:fitted2} Axial Stress-Strain curves of the verification simulations for the fractured granite rock mass under different confining stresses for both the DEM simulations and the fitted CDM simulations.%
}}
\end{center}
\end{figure}

\section{Comparison of CDM Constitutive Models}

\section{Impact of REV Size on Estimated Parameters}

The appropriateness of the REV size was tested using eight different sample REV radii and running the homogenization and parameter estimation algorithms for each. The assumed REV radius for the parameter estimation simulations is $4m$, which corresponds to a homogenization area of $57.7 m^2$. To validate this assumption, the REV radii is sampled at $0.5m$ intervals to see where the resultant parameters converge.

The convergence of three of the 11 parameters is shown in Figure \ref{fig:revconverge} as a function of REV size. The material parameters apparently converge at different sizes, illustrating part of the challenge in defining an REV; some parameters require a larger REV than others and it is not obvious \textit{a priori} which parameters will dominate. For the granite rock mass considered, an REV of radius $3m$ or with a homogenization area of $34.8 m^2$ is chosen to be the minimum size based on the convergence of the dilation angle - the last parameter to converge. The suitability of the assumed REV size is confirmed since it is larger than the minimum REV size determined by the convergence study.

\begin{figure}[!htb]
\begin{center}
\includegraphics[width=\textwidth]{figures/REVAnalysis/REVAnalysis}
\caption{{\label{fig:revconverge} Convergence of three constitutive material parameters as the REV homogenization area is increased. Annotations indicate the specified radius of the circular REV.%
}}
\end{center}
\end{figure}

\section{Comparison to DNS - Application to Slope Stability Analysis}

To validate the up-scaling methodology used, a simple 2-D slope problem is presented and loaded from the top until failure using both DEM and the up-scaled CDM model. Here, the resultant stress distribution are compared just as failure occurs. 

In the DEM model, failure can be assessed based on the unbalanced forces in the model. Since the joints in the model have a stiffness and cohesion, when the slope fails, the explicit quasi-static solution becomes dynamic because of a sudden release of elastic energy and the inability of the applied damping to suppress it all. At this point, the total unbalanced forces in the model increase and the slope can be said to have failed. 

For the CDM model, failure can be assessed based on non-convergence of the model when run as an implicit static simulation, which does not converge when the slope fails. The load step in which the CDM model fails to converge because the slope fails dynamically is considered the point of failure.


\subsection{Model Description}

The plane-strain slope stability problem has a height of $50m$ and a depth of $80m$ with a $30m$ high slope with a grade of $300\%$ (Figure \ref{fig:slopeGeom}). This geometry provides enough space for the failure mechanisms to occur with little influence from the boundaries. The lateral boundary conditions are zero displacement in the x-direction, and the bottom boundary conditions are zero displacement in the y-direction. The slope and top boundaries are free. A uniformly distributed load was applied over a 5m section on the backslope in a linear incremental fashion until failure.


\begin{figure}[!htb]
\begin{center}
\includegraphics[width=0.8\textwidth]{figures/slope diagram1/SlopeSchematic}
\caption{{\label{fig:slopeGeom} Schematic geometry and boundary conditions of the slope failure problem.%
}}
\end{center}
\end{figure}

The meshing of the DEM simulations is identical to the REV simulations for the parameter estimation, while the CDM model is meshed using 4-node bilinear plane strain elements.

The models in both DEM and FEM are allowed to find a static equilibrium after the gravity force is applied, then a linearly increasing compressive stress along the top of the slope is applied until the slope fails. The load is increased from $0MPa$ to $25 MPa$ over the course of $100s$ in the quasi-static/static simulations, knowing that the slope should fail long before $25MPa$ is reached.

\subsection{DNS Comparison}

A qualitative comparison of the DEM and the CDM model results uses the stress distribution and the surface deflection just before failure. For the DEM solution, since the stress field is discontinuous, the stress fields are smoothed using a cubic spline interpolation and subsequently run through a Gaussian filter with a standard deviation of 2 to reduce the noise in the data set. Figures \ref{fig:S11DNS}, \ref{fig:S22DNS}, and \ref{fig:S12DNS} show the horizontal stress distributions, the vertical stress distributions, and the shear stress distributions, respectively.

\begin{figure*}[!htb]
\begin{center}
\includegraphics[width=\textwidth]{figures/S11Contours/S11}
\caption{{\label{fig:S11DNS} Comparison of DEM (left) and CDM (right) horizontal stress contours for the slope just before failure.%
}}
\end{center}
\end{figure*}

\begin{figure*}[!htb]
\begin{center}
\includegraphics[width=\textwidth]{figures/S22Contours/S22}
\caption{{\label{fig:S22DNS} Comparison of DEM (left) and CDM (right) vertical stress contours for the slope just before failure.%
}}
\end{center}
\end{figure*}

\begin{figure*}[!htb]
\begin{center}
\includegraphics[width=\textwidth]{figures/S12Contours/S12}
\caption{{\label{fig:S12DNS} Comparison of DEM (left) and CDM (right) shear stress contours for the slope just before failure.%
}}
\end{center}
\end{figure*}

The continuum approximation of the stress fields shows a good match to the smoothed DEM stress fields. More importantly, the load at failure for the two models are quite close. The DEM simulation failed at $11.2 MPa$, while the CDM simulation failed at $11.5 MPa$, a $~3\%$ error considered to be not only negligible in the context of geological uncertainty but acceptable in terms of the computational savings. This agreement of the two models both in terms of the stress distribution and the failure load shows a high degree of success for the up-scaling framework. 

An additional comparison of the surface deflection where the load was applied is presented in Figure \ref{fig:surfacedeflection}. Again, the behaviour of the two models is similar, with downward displacement occurring where the load is applied, upwards displacement towards the slope on the left and negligible displacement towards the right model boundary. Some divergence from  the DEM results can be observed in the CDM approximation where sharp changes in the profile gradient occur; this arises partly from CDM model limitations and partly because the scale of the deviation is similar to the REV scale, which is the limiting case. For larger scales, the error will be smaller.

\begin{figure*}[!htb]
\begin{center}
\includegraphics[width=\textwidth]{figures/SurfaceDeflection/SurfaceDeflection}
\caption{{\label{fig:surfacedeflection} Comparison of DEM and CDM surface deflection profile for the slope just before failure.%
}}
\end{center}
\end{figure*}

\subsection{Up-Scaling Computational Efficiency}

The CDM model for this case requires about two orders of magnitude less computational effort than the DEM model (Table \ref{tab:computation}). The CDM simulation uses a comparable number of continuum elements ($29,866$) as in the DEM simulation ($25,898$) for comparison and adequate convergence. The CDM model efficiency can be improved by applying a Selectively Refined Mesh (SRM) where only the areas with stress concentrations and large stress gradients have a strongly refined mesh. With the SRM, a converged CDM solution is achievable with only $3,577$ elements leading to another order of magnitude reduction in computational effort. 

\begin{table}[!htb]
\centering
\caption{{Comparison of Computational Time for the DNS}}
\label{tab:computation}
\begin{tabularx}{\textwidth}{@{}YYYYY@{}}
\toprule
\textbf{Simulation Type} & \textbf{Continuum Elements} & \textbf{Processor Clock Speed} & \textbf{Slope Failure Load} & \textbf{Computational Time} \\ \midrule
DEM                      & $25,898$                         & $2.20 GHz$                    & $11.2 MPa$                  & $46.5 hr$                  \\
CDM                      & $29,866$                         & $1.80 GHz$                    & $11.5 MPa$                  & $0.65 hr$                  \\
CDM - SRM                      & $3,577$                         & $1.80 GHz$                    & $11.5 MPa$                  & $0.013 hr$                  \\ \bottomrule
\end{tabularx}
\end{table}

The DEM simulation was run serially on a $2.2GHz$ CPU while the CDM simulation was run serially on a $1.8GHz$ CPU. Despite the CDM model having more continuum elements than the DEM model, and the CDM model running on a slower CPU, a decrease in computational time of the DEM simulation from $46.5 hr$ to $0.65 hr$ was observed. Running the CDM model with a SRM reduces the total computational time to $0.013 hr$, or eight minutes instead of two days. This large increase in computational efficiency with marginal decrease in model accuracy can be immensely useful for large scale geomechanical problems in NFR. 
